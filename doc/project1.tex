\documentclass[english,floatsintext,man]{apa6}

\usepackage{amssymb,amsmath}
\usepackage{ifxetex,ifluatex}
\usepackage{fixltx2e} % provides \textsubscript
\ifnum 0\ifxetex 1\fi\ifluatex 1\fi=0 % if pdftex
  \usepackage[T1]{fontenc}
  \usepackage[utf8]{inputenc}
\else % if luatex or xelatex
  \ifxetex
    \usepackage{mathspec}
    \usepackage{xltxtra,xunicode}
  \else
    \usepackage{fontspec}
  \fi
  \defaultfontfeatures{Mapping=tex-text,Scale=MatchLowercase}
  \newcommand{\euro}{€}
\fi
% use upquote if available, for straight quotes in verbatim environments
\IfFileExists{upquote.sty}{\usepackage{upquote}}{}
% use microtype if available
\IfFileExists{microtype.sty}{\usepackage{microtype}}{}

% Table formatting
\usepackage{longtable, booktabs}
\usepackage{lscape}
% \usepackage[counterclockwise]{rotating}   % Landscape page setup for large tables
\usepackage{multirow}		% Table styling
\usepackage{tabularx}		% Control Column width
\usepackage[flushleft]{threeparttable}	% Allows for three part tables with a specified notes section
\usepackage{threeparttablex}            % Lets threeparttable work with longtable

% Create new environments so endfloat can handle them
% \newenvironment{ltable}
%   {\begin{landscape}\begin{center}\begin{threeparttable}}
%   {\end{threeparttable}\end{center}\end{landscape}}

\newenvironment{lltable}
  {\begin{landscape}\begin{center}\begin{ThreePartTable}}
  {\end{ThreePartTable}\end{center}\end{landscape}}




% The following enables adjusting longtable caption width to table width
% Solution found at http://golatex.de/longtable-mit-caption-so-breit-wie-die-tabelle-t15767.html
\makeatletter
\newcommand\LastLTentrywidth{1em}
\newlength\longtablewidth
\setlength{\longtablewidth}{1in}
\newcommand\getlongtablewidth{%
 \begingroup
  \ifcsname LT@\roman{LT@tables}\endcsname
  \global\longtablewidth=0pt
  \renewcommand\LT@entry[2]{\global\advance\longtablewidth by ##2\relax\gdef\LastLTentrywidth{##2}}%
  \@nameuse{LT@\roman{LT@tables}}%
  \fi
\endgroup}


  \usepackage{graphicx}
  \makeatletter
  \def\maxwidth{\ifdim\Gin@nat@width>\linewidth\linewidth\else\Gin@nat@width\fi}
  \def\maxheight{\ifdim\Gin@nat@height>\textheight\textheight\else\Gin@nat@height\fi}
  \makeatother
  % Scale images if necessary, so that they will not overflow the page
  % margins by default, and it is still possible to overwrite the defaults
  % using explicit options in \includegraphics[width, height, ...]{}
  \setkeys{Gin}{width=\maxwidth,height=\maxheight,keepaspectratio}
\ifxetex
  \usepackage[setpagesize=false, % page size defined by xetex
              unicode=false, % unicode breaks when used with xetex
              xetex]{hyperref}
\else
  \usepackage[unicode=true]{hyperref}
\fi
\hypersetup{breaklinks=true,
            pdfauthor={},
            pdftitle={Project 1: Sea Port Program},
            colorlinks=true,
            citecolor=blue,
            urlcolor=blue,
            linkcolor=black,
            pdfborder={0 0 0}}
\urlstyle{same}  % don't use monospace font for urls

\setlength{\parindent}{0pt}
%\setlength{\parskip}{0pt plus 0pt minus 0pt}

\setlength{\emergencystretch}{3em}  % prevent overfull lines

\ifxetex
  \usepackage{polyglossia}
  \setmainlanguage{}
\else
  \usepackage[english]{babel}
\fi

% Manuscript styling
\captionsetup{font=singlespacing,justification=justified}
\usepackage{csquotes}
\usepackage{upgreek}



\usepackage{tikz} % Variable definition to generate author note

% fix for \tightlist problem in pandoc 1.14
\providecommand{\tightlist}{%
  \setlength{\itemsep}{0pt}\setlength{\parskip}{0pt}}

% Essential manuscript parts
  \title{Project 1: Sea Port Program}

  \shorttitle{Project 1}


  \author{Justin VanWinkle}

  \def\affdep{{""}}%
  \def\affcity{{""}}%

  \affiliation{
    \vspace{0.5cm}
          \textsuperscript{} 26 March 2017\\
          \textsuperscript{} CMSC 335 (7980)  }

 % If no author_note is defined give only author information if available
    \authornote{
    \newcounter{author}
                      Correspondence concerning this article should be addressed to Justin VanWinkle, 281 4th St, Geneva, FL 32732. E-mail: \href{mailto:justin@swissarmydev.com}{\nolinkurl{justin@swissarmydev.com}}
                }
  

  \abstract{This paper is dedicated to a project developed in a course on
object-oriented programming and concurrency. It explores the decisions
made while designing the program, the way a user should go about making
use of the program, standard practices for testing the program, lessons
learned while}
  \keywords{java,object-oriented \\

    
  }





\begin{document}

\maketitle

\setcounter{secnumdepth}{0}



\section{Design}\label{design}

From a high level, this program has been built with a class that
contains the UI separate from any workflow logic. This paves the way for
several simplistic approaches when concurrency is included into the
program in a future iteration. One such example would be creating the
world using a thread that is separate from that of the Event Dispatch
Thread that would be used by default when calling a function from an
event listener that is attached to a swing component. Such a separation
becomes easily manageable when business logic and view logic remain
isolated.

\subsection{Decisions}\label{decisions}

\begin{itemize}
\tightlist
\item
  It was determined that all classes that are representative of objects
  in the input file would not instantiate a scanner so as to prevent
  repetitive code across classes and also to prevent each class from
  needing to know how its definition is formed in the input file. This
  loosens the coupling in such a way that eases changes as a result of
  changes in the structure of the input file.
\item
  TDD was used for some areas of the program to ensure that changes in
  future iterations do not alter the functionality of those methods.
\item
  A map was used in the World class to track all objects for two main
  reasons. First, the complexity of creating the model is greatly
  reduced since parent objects can be found without having to search for
  them. Second, this allows for O(n) lookup time for any object that
  exists in the World. So for the purposes of this class, the time
  needed to create the World is drastically reduced -- especially for
  very large input files.
\end{itemize}

\subsection{Meanings}\label{meanings}

The various classes, packages, variables, and methods in this program
are organized in a logical manner that show good practices in
object-oriented design. Each of the items has been placed in such a way
that it has some sort of direct relationship with its containing parent
or its contained child.

\subsubsection{Classes}\label{classes}

All classes assume no knowledge of their calling class whether by
passing context or interface. This is likely to change in a future
iteration, however.

\paragraph{\texorpdfstring{The \emph{Thing}
Package}{The Thing Package}}\label{the-thing-package}

Each class in the \emph{thing} package as well as classes in the sub
packages of \emph{thing} represent real world objects that reflect their
name.

\paragraph{Other Classes}\label{other-classes}

The PortTime class, in a similar manner represents the time of a port.
Due to ambiguous requirements, however, the developer was unable to
determine the intended use of this class. Therefore, it is assumed that
this class will be better integrated in a future iteration.

\subsubsection{Variables}\label{variables}

In each class, there are sets of instance variables that represent a
\emph{has-a} relationship between the class and the item that variable
holds. For instance, a SeaPort has a (or many) Dock(s).

\subsubsection{Methods}\label{methods}

The SeaPortProgram class contains the heart of creating the world
including all of the logic for parsing out the input file, creating the
objects, and modeling a world that represents what is defined in the
input file. While the requirement documentation \emph{suggests} that a
scanner should be used, a string tokenizer was used instead. While this
adds some additional coding overhead,

\section{User's Guide}\label{users-guide}

This serves as a guide for all you need to know to get up and running
with this program

\subsection{How to start the program}\label{how-to-start-the-program}

To compile, cd into the src directory and run the command
\texttt{javac\ *.java\ */*.java\ */*/*.java}. Once that command has
completed, run
\texttt{java\ -classpath\ \textless{}path-to-src-folder\textgreater{}\ main.SeaPortProgram}.

\subsection{How to create a world}\label{how-to-create-a-world}

\begin{enumerate}
\def\labelenumi{\arabic{enumi}.}
\tightlist
\item
  Click \enquote{Select File}.
\item
  Navigate to and select a valid input file.
\item
  Select \enquote{Open}.
\item
  World is automatically modeled. Notice the output that matches the
  data in the chosen input file.
\end{enumerate}

\subsection{Search}\label{search}

Searches using regex provide the user with more control over the
matching pattern. Matches look at three criteria: name, index, and
skill.

\subsubsection{How to search}\label{how-to-search}

\begin{enumerate}
\def\labelenumi{\arabic{enumi}.}
\tightlist
\item
  Input a valid java regex engine pattern in the search box. (Valid
  regex patterns for the java regex engine can be obtained
  \href{http://docs.oracle.com/javase/8/docs/api/java/util/regex/Pattern.html}{here}.
\item
  Click \enquote{Search}. Searching returns a list of all \emph{Things}
  that have a Name or Index that match the criteria. Some additional
  cases exist such as searching the skills of a Person or the duration
  of a Job.
\end{enumerate}

\subsubsection{Examples}\label{examples}

\begin{itemize}
\tightlist
\item
  Search Pattern: \enquote{ar}

  \begin{itemize}
  \tightlist
  \item
    Matches \enquote{Sara}
  \item
    Does not Match \enquote{Archie}
  \end{itemize}
\item
  Search Pattern: \enquote{{[}A\textbar{}a{]}r}

  \begin{itemize}
  \tightlist
  \item
    Matches \enquote{Sara} and \enquote{Archie} For more extended
    documentation on regex patterns,
    \href{http://docs.oracle.com/javase/8/docs/api/java/util/regex/Pattern.html}{go
    here}.
  \end{itemize}
\end{itemize}

\subsection{Special Features}\label{special-features}

Nothing notable.

\section{Test Plan}\label{test-plan}

\begin{enumerate}
\def\labelenumi{\arabic{enumi}.}
\tightlist
\item
  Base Case

  \begin{itemize}
  \tightlist
  \item
    \textbf{Input:} (1.txt) Using the input file outlined in the
    requirements documentation
  \item
    \textbf{Expected:} Ouput that matches the output outlined in the
    documentation.
  \item
    \textbf{Result:} An exact match between the expected output and the
    actual output.
  \end{itemize}
\item
  Very large input file

  \begin{itemize}
  \tightlist
  \item
    \textbf{Input:} (2.txt) Using an input file that is extremely large
    to ensure no allocation issues
  \item
    \textbf{Expected:} No errors and output that displays an appropriate
    data structure to match the input.
  \item
    \textbf{Result:} No errors thrown and output matches an appropriate
    construction of input.
  \end{itemize}
\item
  Unit tests

  \begin{itemize}
  \tightlist
  \item
    Unit tests were written to ensure that units of code perform the
    same basic function even after changes are made.
  \item
    \textbf{Expected:} All unit tests should pass.
  \item
    \textbf{Result:} All unit tests pass.
  \end{itemize}
\end{enumerate}

\section{Lessons Learned}\label{lessons-learned}

In the coming sections, consideration will be given to particular pain
points, gleanings, and takeaways.

\subsection{About This Documentation}\label{about-this-documentation}

Since discovering Pandoc, I have written all of my professional
documentation and papers in markdown and used Pandoc to convert them to
whatever end format I require. Markdown allows me to focus on content
rather than formatting and thus saves great amounts of time. One
particular hurdle in writing this documentation was finding a way to use
\LaTeX~to handle the formatting in a way that was consistent with the
APA style guide. As a result, I came across a set of tools that allows
me to continue using Pandoc to pipe everything through a template that
\LaTeX~uses to ultimately format my paper perfectly according to the APA
style -- and all I ever did was write this in markdown with some yaml
front matter. A glorious combination when paired with Vim.

\subsection{New Classes Used}\label{new-classes-used}

While I purposefully avoid Java's built in UI libraries, it came as no
surprise when I found myself using the JScrollPane class for the first
time in this program. I did find JScrollPane to be comparatively easy to
work with for the sake of just getting a scroll view set up.

\subsection{Future Iterations}\label{future-iterations}

In a future iteration, I would like to focus on pushing all UI driven
events to the EDT and all other tasks onto other threads. In the words
of Vern Gosdin, \enquote{This ain't my first rodeo.} Additionally, I
would likely opt for some sort of utility class if too much begins to be
repeated among the various classes. I already see where this might be
possible as the search expands and I could offload the matching to a
static utility method that just grabs the specified fields from each of
the classes for comparison. Thing implements the comparable interface,
but makes no actual use of the interface. Currently, the compareTo()
method is overridden in Thing, but in the future, Thing need to be an
abstract class and all subclasses will need to override this comparable
method. Lastly, there is plenty of room for additional unit tests. New
unit tests should be written on any logic code before any alterations
are made for concurrency to ensure proper results.

\begin{figure}[htbp]
\centering
\includegraphics{/Users/vanwinklej/workspace/UMUC/oop-concurrency/project/doc/diagram.png}
\caption{UML Diagram}
\end{figure}

\setlength{\parindent}{-0.5in} \setlength{\leftskip}{0.5in}






\end{document}
